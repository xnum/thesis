\begin{titlepage}
  \begin{center}
  	\LARGE 
    \begin{singlespace}    
      \textbf{\chineseTitle{}} \\[0.5cm]
    \end{singlespace}
    
    \begin{singlespace}    
    \begin{tabular}{r l}
    	學生     & :\studentCnName{}  \\
        指導教授  & :\advisorCnName{} \hspace{0.1cm} 教授 \\[0.5cm]
    \end{tabular}
    \end{singlespace}

    國立交通大學資訊科學與工程研究所碩士班 \\[0.5cm]
    \makebox[4em][s]{摘要} \\[0.5cm]
    	
  \end{center}
  \normalsize 
  \hspace{0.6cm} 在程式自動化測試與分析上,符號執行(symbolic execution)是目前經常被使用的一種方法,由於符號執行會紀錄並模擬出程式執行時的所有可能路徑,其數量會以指數的數量級不停成長,最終耗盡所有運算資源,這個問題被稱為路徑爆炸問題(path explosion problem);因此我們需要在有限的資源內採取某些策略來優先模擬較有價值的路徑,在本篇論文中我們提出使用以蒙地卡羅搜尋樹為基礎的搜尋策略來解決這個問題,並比較它與其他傳統策略如深度優先搜尋、廣度優先搜尋的效率。
  \\[0.7cm]
  關鍵字:Monte Carlo Tree Search(MCTS), Upper Confidence Bounds for Trees (UCT), symbolic execution
\end{titlepage}