\documentclass[12pt,a4paper,oneside]{book}

\usepackage{booktabs}
\usepackage{algorithm}
\usepackage{algpseudocode}
%\usepackage{algorithmic}
\usepackage{amsmath}
\usepackage{graphics}
\usepackage{epsfig}
% border setting
\usepackage[ top=2.5cm,bottom=2.5cm,left=2.5cm,right=2.5cm ]{ geometry }

% The default for LaTeX is to have no indent after sectional headings, like \chapter and \section. ()
\usepackage{indentfirst}

% http://tex.stackexchange.com/questions/28333/continuous-v-per-chapter-section-numbering-of-figures-tables-and-other-docume
\usepackage{chngcntr}
\counterwithout{figure}{chapter}
\counterwithout{table}{chapter}

% This prevents placing floats before a section
\usepackage[section]{placeins}
\let\Oldsubsection\subsection
\renewcommand{\subsection}{\FloatBarrier\Oldsubsection}

% source code hightlighting
\usepackage{listings}
\lstset{
  numbers=left,
  stepnumber=1,
  firstnumber=1,
  captionpos=b,
  tabsize=2,
  basicstyle=\small,
  numberfirstline=true
}

% setting the page number to footer
\usepackage{fancyhdr}
\fancyhf{}
\cfoot{\thepage}
\pagestyle{fancy}
% no header and footer bar
\renewcommand{\headrulewidth}{0pt}
\renewcommand{\footrulewidth}{0pt}

% setup bibliography
\usepackage[sorting=none,backend=bibtex]{biblatex}
\addbibresource{reference.bib}

% line height setting
\linespread{1.5}
\usepackage{setspace}

% Graphics settings
\usepackage{graphicx}
\graphicspath{ {./figures/} }

\usepackage{background}
\newcommand\DeactivateBG{\backgroundsetup{contents={}}}
\newcommand\ActivateBG{
  \backgroundsetup{
      contents={\includegraphics[]{logo.jpg}},
      scale=1,
      opacity=0.4,
      angle=0
  }
}

\usepackage{comment}

% do not place figure at the middle of a empty page
\makeatletter
\setlength{\@fptop}{0pt}
\makeatother

% Chinese typesetting
\usepackage{xeCJK}
%\setCJKmainfont{SourceHanSansTW-Light.otf}
\setCJKmainfont[
  BoldFont={SourceHanSansTW-Normal.otf},
  ItalicFont={SourceHanSansTW-Light.otf}
]{SourceHanSansTW-Light.otf}
\newcommand{\myHuge}[1]{\fontsize{40}{50} #1}

\newcommand{\chineseTitle}{基於蒙地卡羅樹搜尋法之路徑探索於符號執行}
\newcommand{\englishTitle}{MCTS-based Path Exploration for Symbolic Execution}

\newcommand{\studentCnName}{葉家郡}
\newcommand{\studentEnName}{Jia-Jun Yeh}
\newcommand{\advisorCnName}{黃世昆}
\newcommand{\advisorEnName}{Shih-Kun Huang}

\usepackage{pdfpages}
\DeclareMathOperator*{\argmax}{arg\,max}
\begin{document}

\pagenumbering{gobble} % disabling page numbering

\ActivateBG

\input{cover1.tex}
\input{cover2.tex}

\DeactivateBG

% 放審定書和授權書
% \includepdf[pages={1-2}]{auth.pdf}

\ActivateBG

\begin{titlepage}
  \begin{center}
  	\LARGE 
    \begin{singlespace}    
      \textbf{\chineseTitle{}} \\[0.5cm]
    \end{singlespace}
    
    \begin{singlespace}    
    \begin{tabular}{r l}
    	學生     & :\studentCnName{}  \\
        指導教授  & :\advisorCnName{} \hspace{0.1cm} 教授 \\[0.5cm]
    \end{tabular}
    \end{singlespace}

    國立交通大學資訊科學與工程研究所碩士班 \\[0.5cm]
    \makebox[4em][s]{摘要} \\[0.5cm]
    	
  \end{center}
  \normalsize 
  \hspace{0.6cm} 在程式自動化測試與分析上,符號執行(symbolic execution)是目前經常被使用的一種方法,由於符號執行會紀錄並模擬出程式執行時的所有可能路徑,其數量會以指數的數量級不停成長,最終耗盡所有運算資源,這個問題被稱為路徑爆炸問題(path explosion problem);因此我們需要在有限的資源內採取某些策略來優先模擬較有價值的路徑,在本篇論文中我們提出使用以蒙地卡羅搜尋樹為基礎的搜尋策略來解決這個問題,並比較它與其他傳統策略如深度優先搜尋、廣度優先搜尋的效率。
  \\[0.7cm]
  關鍵字:Monte Carlo Tree Search(MCTS), Upper Confidence Bounds for Trees (UCT), symbolic execution
\end{titlepage}
\begin{titlepage}
  \begin{center}
    \LARGE
    \begin{singlespace}
  	 \textbf{\englishTitle{}} \\[0.5cm]
    \end{singlespace}
    
    \begin{singlespace}
    \begin{tabular}{r l}
    	Student     & : \studentEnName{}  \\
        Advisor  & : Dr. \advisorEnName{} \\[0.5cm]
    \end{tabular}
    \end{singlespace}
	
    \begin{singlespace}
    Institute of Computer Science and Engineering National Chiao Tung University\\[0.5cm]
    \end{singlespace}
    \textbf{ABSTRACT} \\[0.5cm]
    	
  \end{center}
  \normalsize 
  \hspace{0.6cm} Symbolic execution is a technology that is often used today in program automation testing and analysis. Since symbol execution traces and simulates all possible paths when a program executes, its number grows exponentially. This problem is called the path explosion problem. Therefore, we need to take some strategies within the limited resources to give priority to the more valuable path. In this paper, we propose to use The Carlo search tree-based search strategy solves this problem and compares it with other classic strategies such as depth-first search (DFS) and breadth-first search (BFS).
  \\[0.7cm]
  Keywords:Monte Carlo Tree Search(MCTS), Upper Confidence Bounds for Trees (UCT), symbolic execution
\end{titlepage}

\tableofcontents
\listoffigures
\listoftables

% 9/30
\chapter{Introduction} \pagenumbering{arabic} % enabling page numbering

在現今的軟體開發中,程式碼數量動輒數萬行,系統的複雜度也越來越高,傳統驗證程式正確性的方式如unit testing、code review等等...受限於人工而相當有限;在硬體計算能力突飛猛進的現代,自動化測試的方式又逐漸成為顯學,如fuzzing、symbolic execution...能夠自動尋找程式中可能的漏洞,其中symbolic execution是一種模擬執行的方法,它將程式的使用者輸入視為符號,並把程式執行過程和分支條件轉換為限制式,藉由求解限制式來獲得欲執行該路徑所需的使用者輸入為何;由於symbolic execution在遇到分支時會複製出一條新的路徑,兩條路徑分別探索執行if時和執行else時的狀態,因此路徑數量會以指數的數量級成長,造成探索路徑需要花費巨大的運算資源,這個問題被稱為path explosion problem,這篇論文欲以蒙地卡羅樹搜尋演算法來找出探索價值較高的路徑,使得symbolic execution能使用較短的時間內獲得更高的程式執行覆蓋率。

蒙地卡羅樹搜尋(Monte Carlo tree search)被廣泛的運用在遊戲人工智慧中,例如西洋棋、黑白棋、圍棋等等的棋盤遊戲,在2016年AlphaGo(一個結合蒙地卡羅樹搜尋和深度學習的圍棋AI程式)擊敗世界棋王後,更是一度引起相當多的討論;此演算法主要的精神在於對一棵樹,選擇子結點並嘗試用模擬的方式估計該結點的價值。和symbolic execution相當類似的地方在於同為需要探索一棵樹,而且要避免探索一些我們不感興趣或是沒有必要探索的路徑,例如無窮迴圈的狀況;我們認為結合蒙地卡羅樹搜尋做為symbolic execution挑選路徑執行的演算法,相較於傳統的深度優先搜尋法或廣度優先搜尋法,能有較佳的搜尋效率。

% \cite{thesis}.

% 9/30
\chapter{Background}

在這個章節將簡要介紹Symbolic execution與其遇到的問題,還有蒙地卡羅樹搜尋演算法的流程和優點。

\section{Symbolic execution}

% 說明symbolic execution的流程和精神
在本篇論文中所提及的Symbolic execution為Dynamic symbolic execution(又名為Concolic execution),首先由K. Sen\cite{sen2007concolic}提出,和基於其想法實作的程式DART\cite{godefroid2005dart}、CUTE\cite{sen2005cute},而其近年又分為兩種類型,需要程式原始碼的code-based symbolic execution,如KLEE\cite{cadar2008klee}和不須程式碼而直接分析程式執行檔的binary-based symbolic execution如S2E\cite{chipounov2012s2e}、Mayhem\cite{cha2012mayhem};symbolic execution engines在分析程式時會將其載入並先轉換為intermediate representation (IR),接著會將使用者輸入(例如標準輸入、檔案、命令列參數)標記為symbolic變數,接著模擬程式的執行過程,將執行過程中遇到的程式碼轉換為數學邏輯限制式的形式,當在模擬的過程中遇到分支條件時,便複製出一條新的路徑,分別追蹤該分支條件為真和為假時的情況;當模擬執行結束時,把先前蒐集的邏輯限制式利用solver求解(如:SMT\cite{vanegue2012smt}、Z3\cite{Z3}等等),以取得欲執行該路徑所需的實際輸入值;透過這個流程理論上我們可以追蹤所有的執行路徑,探索是否有不當的輸入值能觸發程式崩潰。

\section{Path Exploration}

% 說明symbolic execution中的路徑探索問題
雖然Symbolic execution理論上能夠探索所有執行路徑,但在探索的過程中可能會遭遇到一些問題:當分支條件取決於symbolic變數時,很有可能兩邊的條件都有可能成立,這意謂著程式必須複製並分別維護兩條路徑,直到發現該路徑無解為止,當路徑被不斷的複製就產生了路徑爆炸問題(path explosion problem)。

\section{Monte Carlo tree search}

MCTS是一種啟發式搜尋演算法,近年來最廣為人知的應用是遊戲AI方面的演算法;Monte Carlo模擬是利用模擬和統計,得到一個近似解,在足夠大量的模擬下,理論上我們可以得到一個跟最佳解非常接近的答案。MCTS套用了這種模擬的方式,維護一棵樹(在遊戲AI中通常是一個遊戲盤面的狀態樹)並統計每個盤面的勝率,期望能只探索部分的樹,而非全部探索完的情況下,就能知道該盤面的勝率。

在\cite{browne2012surveyMCTS}中說明了MCTS演算法的基本流程,如Figure \ref{figMCTS}:

\begin{figure}[h]
\center
\includegraphics[width=\textwidth,height=\textheight,keepaspectratio]{figures/mcts2.PNG}
\caption{Monte Carlo Tree Search \label{figMCTS}}
\end{figure}

\begin{itemize}
\item \textbf{Selection} 根據設定的\textit{Tree Policy},從根節點開始遞迴性的決定一個目前最需要展開的節點,如Figure \ref{figMCTS}中的Selection部分,以粗框標記出的節點。可展開的節點在這裡定義為狀態尚未終止且有尚未訪問的子節點。
\item \textbf{Expansion} 在選擇的節點上,執行一個合法的動作來新增子節點,如Figure \ref{figMCTS}中的Expansion部分,在挑選的節點下新增一個節點。
\item \textbf{Simulation} 從這個新增的節點上使用\textit{Default Policy}來進行模擬執行,產生結果。
\item \textbf{Backpropagation} 模擬的結果會回饋到步驟1所選擇路過的那些節點上,更新他們的統計數據。
\end{itemize}

在這邊有兩個policy:\textit{Tree Policy}代表的是如何決定要選擇和新增節點的演算法。\textit{Default Policy}則是從該節點的狀態模擬對局直到獲得勝負結果。雖然這兩個Policy也可以簡單的使用隨機方式決定,但適當的演算法有助於強化MCTS的準確度,如\cite{Intro2MCTS}便指出,Tree Policy可以使用upper confidence bound (UCB)演算法來取代隨機挑選,當可能的選擇有很多,卻只有少數一兩個有可能被選擇時,隨機選擇不太容易挑到,而造成模擬的結果不精準,如果我們把盤面的位置當成吃角子老虎機,視為multi-armed bandit problem來處理的話,會比原本的隨機選擇好;另外他也指出模擬時除了用這些簡單的方法,也可以使用更耗費資源的啟發式邏輯和評價方式,在對於higher branching factor的遊戲會有較好的效果。

\chapter{Design}

MCTS演算法是為了要在特定時間或電腦資源使用量內做出某個決策,利用隨機模擬與統計的方式尋找出一個最佳解,如何最有效的擴展空間狀態樹是我們的演算法可以仿效之處。對symbolic execution而言,目標就變成如何達到更高的執行覆蓋率,以盡量的挖掘出可能的漏洞。

我們希望在套用MCTS演算法後,能延緩path explosion problem的發生,並使symbolic execution相較於傳統的搜尋方法能在同樣的資源限制下(如時間限制、記憶體限制)走訪更多的程式碼。

\begin{algorithm}[htbp]
  \caption{applying UCT algorithm to symbolic execution}
  \begin{algorithmic}[1]
  	\Function{Search}{$p_r$}
    \State set $p_r$ as root of Tree $T$
    \State $B \leftarrow \emptyset$
    \While {within computational budget}
      \State $p$ $\leftarrow$ TreePolicy($T$)
      \State $B \leftarrow B \bigcup p$
      \State $S$ $\leftarrow$ step($p$)
      \For{each path $p_c \in S$}
      	\State $V \leftarrow$ DefaultPolicy($p_c$)
        \State $Q(p_c) \leftarrow \alpha \frac{|V-B|}{N} + \beta|p_c|$
        \State add a new child $p_c$ to $p$
      \EndFor
      \State BackPropagation($p$)
    \EndWhile
    \EndFunction
  \end{algorithmic}
\end{algorithm}

Algorithm 1 為我們設計的演算法主體。$p_r$為現在要搜尋的路徑,首先將$p_r$設為樹$T$的root,以記錄路徑間的關係,並以集合$B$來計算路徑走訪的數量。在第5行我們會先利用\textit{TreePolicy}從樹$T$中挑選一個應該計算的路徑$p$,並將$p$加進集合$B$中(路徑$p$事實上可以被視為是數個走訪過的位址集合),接著將$p$遞交給symbolic engine進行計算,將這條路徑step一步;step後可能產生數條路徑,我們以集合$S$來表示,$S$中的路徑們代表$p$ step後可能的狀態,如執行if時或執行else時的狀態,遇到switch case時執行不同case的狀態;在第8行對$S$中的每條路徑$p_c$我們都會利用$DefaultPolicy$來模擬其未來可能的走向,並評估此路徑的價值$Q(p_c)$,並將$p_c$標記為$p$的child;最後在第13行整理先前第8-12行的資訊並記錄起來。

在第10行為我們計算路徑價值的公式,此公式分為兩個部分,其中$\alpha$、$\beta$和$N$為可以人為控制的參數。$\frac{|V-B|}{N}$中的$V$為路徑$p_c$經由\textit{DefaultPolicy}計算後得出未來可能會執行的位址集合,和集合$B$取差集後計算其數量,除以$N$(\textit{DefaultPolicy}模擬的次數)就是路徑$p_c$增加程式執行區塊覆蓋率的期望值;而$|p_c|$為$p_c$執行過的程式碼區塊數量,這個參數是為了在所有路徑的覆蓋率期望值都非常低的時候,能優先選擇較深的路徑避免進行無謂的探索。$\alpha$和$\beta$是這兩個數值的權重。計算$Q$時的$\alpha$主要控制的是增加覆蓋率的期望值,由於路徑的模擬是根據Control flow graph(CFG)來猜測,如果產生的CFG不正確或程式的實際執行狀況和模擬的結果有落差,期望值就會變得不準確,相對的對於簡單的小程式,模擬的準確度有可能是較高的,因此適當的調整$\alpha$可以修正模擬的數據;而$\beta$是為了因應遇到大量迴圈或strcmp這類函式的措施,由於進入迴圈容易產生大量的分枝,會讓path數量一下子成長很多,\textit{TreePolicy}在選擇時也容易被混淆,我們透過計算該path已經執行過的程式碼區塊數量,讓演算法在挑選時偏好已經執行比較多區塊數量的路徑。

\begin{algorithm}[htbp]
  \caption{Policies for our algorithm}
  \begin{algorithmic}[]
    \Function{TreePolicy}{$T$}
    	\State $n \leftarrow$ root of $T$
        \While {$n$ is not terminated}
        	\If{$n$ is expandable}
            	\State \Return {$n$}
            \Else
            	\State $n \leftarrow BestChild(n,C)$
            \EndIf
        \EndWhile
    \EndFunction
    \item[]
	\Function{DefaultPolicy}{$p$}
    	\State $V \leftarrow \emptyset$
        \For{$i=1$; $i<N$; $i++$}
          \State $v \leftarrow$ find vertex at CFG($p$'s addr)
          \For{$j=1$; ($j<M$)and($v$ has any edge); $j++$}
              \State add $v$ to $V$
              \State $v \leftarrow$ random pick a vertex which $v$ directed to
          \EndFor
        \EndFor
        \State \Return $V$
    \EndFunction
    \item[]
    \Function{BestChild}{$p,C$}
    	\State $Q_{max} \leftarrow \operatorname*{arg\,max}_{p_c \in p} Q(p_c)$
    	\State \[ \Return \argmax_{p_c \in p} \frac{Q(p_c)}{Q_{max}}+C\sqrt[]{\frac{2\ln N(p)}{N(p_c)}} \]
    \EndFunction
    \item[]
    \Function{BackPropagation}{$v$}
    \While{$v$ is not null}
    	\State N($v$) += 1
		\State Q($v$) $\leftarrow$ average of Q($v$'s children)
    \State $v \leftarrow$ parent of $v$
    \EndWhile
    \EndFunction
  \end{algorithmic}
\end{algorithm}

Algorithm 2是Algorithm 1中所提及的函式實作部分。\textit{TreePolicy}用來挑選應該被計算的路徑,而價值計算事實上是由$BestChild$決定,其中的一個參數$C$,其影響的是MCTS中常被提及的特徵:expolitation和exploration,也就是程式該往較深的點進行計算,還是選擇較少被計算過的點,不過這個數值在branching factor高時較有效,而我們產生出的path常常只有1至2個,所以這個參數的影響並不大,唯一會影響的是當某個節點尚未被計算過任何一次的時候,$N(p_c)$為0,根號內的數值會是無限大,程式就一定會選擇該節點來進行計算。\textit{DefaultPolicy}的參數$N,M$,$N$是要進行幾次模擬,而$M$是避免程式進入無窮迴圈,當到達一定數字時會強制中斷模擬,對於較小的程式可以選擇較小的數字,而較複雜的程式可以選擇比較大的數字來增加其準確性,但也相對的花費更多時間。$BackPropagation$則是更新兩項數值,$N(v)$為$v$被訪問的次數,對於價值$Q(v)$則以v的子節點們的平均來替代。

\chapter{Evaluation}

\section{實驗環境與方法}

我們將本篇論文提出的演算法實作於Angr (一個開源的python符號執行框架) \cite{angr}上,並在Ubuntu 16.04作業系統上安裝與執行所有需要的程式和測試的目標程式,目標程式的名稱與版本如下Table. \ref{binarys}。硬體環境使用的是Intel i7-2600k處理器以及24 GB記憶體。

\newcommand{\ra}[1]{\renewcommand{\arraystretch}{#1}}
\begin{table*}[h]\centering
\ra{1.3}
\label{binarys}
\caption{Target Program's name and version}
\begin{tabular}{@{}ll@{}}\toprule
program name & version \\ \midrule
cp           & 8.25    \\ 
echo         & 8.25    \\ 
hostname     & 3.16    \\ 
ls           & 8.25    \\ 
mkdir        & 8.25    \\ 
ps           & 3.3.10  \\ 
readelf      & 2.26.1  \\ 
touch        & 8.25    \\ 
cpp-markdown & 1.00    \\ 
gif2png      & 2.5.8   \\ \bottomrule
\end{tabular}
\end{table*}

\section{常數對演算法的效率影響}



\section{演算法與其他方法的比較}

\section{長時間執行的效率比較}

\chapter{Related Work}

\section{Path Selection Problem}

在\cite{sharma2012critical}\cite{schwartz2010all}中對symbolic execution所做的概述,提及了幾項目前的挑戰,其中最大的問題就是路徑選擇問題(path selection problem),當路徑爆炸問題已成一個不可避免的問題,目前已經提出的幾種解決方式如:KLEE\cite{cadar2008klee}中提出的深度優先搜尋法來優先探索最深的路徑,中便使用這種作法,但很有可能被困在無限迴圈而進行了無效的探索。\cite{sen2007concolic}提出的concolic testing,除了以symbolic execution執行,也會使用具體的輸入真正執行程式來蒐集執行路徑,\cite{sen2005cute}就使用了這種作法。而KLEE\cite{cadar2008klee}也支援隨機挑選路徑的方式來避免進行了無效的探索。

另外如driller\cite{stephens2016driller}結合了AFL\cite{AFL}的fuzzing技術,它將使用者輸入分類為需要特定值的特定輸入(specific input)和可接受各種數值的通用輸入(general input),並透過symbolic execution engine和fuzzing engine的切換來解決各自不擅長的部分。而s2e\cite{chipounov2012s2e}則是利用選擇性的symbolic execution,避免連同其他函式庫也一起分析,造成路徑數量大量增長。另外也有針對路徑成長,檢查其可滿足性(satisfiability)並做動態剪枝的方法\cite{PathPruning}。

\section{MCTS and Game AI}

Monte Carlo方法是一種隨機取樣的方法,在1987年由Bruce Abramson提出\cite{mcmethod}。而在1989年,Monte Carlo tree search由W. Ertel, J. Schumann和C. Suttner提出,用來改善搜尋演算法的時間如DFS、BFS等等。而在1993年B. Brügmann首先將Monte Carlo方法用於圍棋上\cite{mc_go},直到2006年這個方法才由Rémi Coulom真正被命名為Monte Carlo tree search\cite{MCTS_naming}。之後 L. Kocsis and Cs. Szepesvári以MCTS為基礎發表了upper confidence bound 1 applied to trees (UCT)演算法\cite {UCT}。在2015年由Google Deepmind研發的圍棋AI AlphaGo\cite{alphago},使用MCTS和deep learning演算法,擊敗了人類職業選手,頓時之間AI和機器學習又成為電腦科學界的顯學。

\chapter{Conclusion}

我們提出的演算法在實驗中證實,在相同的時間限制和資源限制下,比起傳統的DFS和BFS策略能有效的探訪更多程式碼,同樣在沒有修改任何程式或資料的情況下,我們額外使用一棵樹來記錄路徑的資訊和親子關係,就算沒有使用MCTS的方法,紀錄路徑間的親子關係也有可能用來作為其他演算法上的cut或pruning使用;唯一被修改的是路徑進入symbolic engine計算的順序,這使我們的方法與其他技術如concolic testing\cite{sen2007concolic}, veritesting\cite{Veritesting}, driller\cite{stephens2016driller}整合的難度有可能是變低的。

但我們的演算法仍然可能會有以下的問題,第一是path explosion problem,如果我們完全不修剪路徑,它仍然可能發生,只能盡力在發生前挑選價值高的路徑進行搜索;如果借助我們建立的樹來進行修剪,雖然可以延緩甚至避免這個問題發生,但有些路徑可能就不會被探索到。第二是symbolic execution既有的問題,當它遇到大量迴圈如strcmp函式,會產生大量路徑,我們的演算法也會遇到這個問題,這只能依靠veritesting或fuzzing來獲得解決。最後是CFG的問題,當靜態分析發生錯誤,如產生出的CFG有部分不正確,或是被混淆代碼(obfuscated code)等等可能的因素,那演算法對於模擬計算出的期望值就可能沒有效果,這時就只能依靠路徑的已執行程式碼區塊數量來判斷,演算法的效果會變差。

\newpage

\printbibliography[title={References}]

\end{document}
