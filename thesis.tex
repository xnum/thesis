\documentclass[12pt,a4paper,oneside]{book}

% border setting
\usepackage[ top=2.5cm,bottom=2.5cm,left=2.5cm,right=2.5cm ]{ geometry }

% The default for LaTeX is to have no indent after sectional headings, like \chapter and \section. ()
\usepackage{indentfirst}

% http://tex.stackexchange.com/questions/28333/continuous-v-per-chapter-section-numbering-of-figures-tables-and-other-docume
\usepackage{chngcntr}
\counterwithout{figure}{chapter}
\counterwithout{table}{chapter}

% This prevents placing floats before a section
\usepackage[section]{placeins}
\let\Oldsubsection\subsection
\renewcommand{\subsection}{\FloatBarrier\Oldsubsection}

% source code hightlighting
\usepackage{listings}
\lstset{
  numbers=left,
  stepnumber=1,
  firstnumber=1,
  captionpos=b,
  tabsize=2,
  basicstyle=\small,
  numberfirstline=true
}

% setting the page number to footer
\usepackage{fancyhdr}
\fancyhf{}
\cfoot{\thepage}
\pagestyle{fancy}
% no header and footer bar
\renewcommand{\headrulewidth}{0pt}
\renewcommand{\footrulewidth}{0pt}

% setup bibliography
\usepackage[sorting=none,backend=bibtex]{biblatex}
\addbibresource{reference.bib}

% line height setting
\linespread{1.5}
\usepackage{setspace}

% Graphics settings
\usepackage{graphicx}
\graphicspath{ {./figures/} }

\usepackage{background}
\newcommand\DeactivateBG{\backgroundsetup{contents={}}}
\newcommand\ActivateBG{
  \backgroundsetup{
      contents={\includegraphics[]{logo.jpg}},
      scale=1,
      opacity=0.4,
      angle=0
  }
}



% do not place figure at the middle of a empty page
\makeatletter
\setlength{\@fptop}{0pt}
\makeatother

% Chinese typesetting
\usepackage{xeCJK}
%\setCJKmainfont{SourceHanSansTW-Light.otf}
\setCJKmainfont[
  BoldFont={SourceHanSansTW-Normal.otf},
  ItalicFont={SourceHanSansTW-Light.otf}
]{SourceHanSansTW-Light.otf}
\newcommand{\myHuge}[1]{\fontsize{40}{50} #1}

\newcommand{\chineseTitle}{中文題目}
\newcommand{\englishTitle}{英文題目}

\newcommand{\studentCnName}{葉家郡}
\newcommand{\studentEnName}{Jia-Jun Yeh}
\newcommand{\advisorCnName}{黃世昆}
\newcommand{\advisorEnName}{aaa}

\usepackage{pdfpages}

\begin{document}

\pagenumbering{gobble} % disabling page numbering

\ActivateBG

\input{cover1.tex}
\input{cover2.tex}

\DeactivateBG

% 放審定書和授權書
% \includepdf[pages={1-2}]{auth.pdf}

\ActivateBG

\begin{titlepage}
  \begin{center}
  	\LARGE 
    \begin{singlespace}    
      \textbf{\chineseTitle{}} \\[0.5cm]
    \end{singlespace}
    
    \begin{singlespace}    
    \begin{tabular}{r l}
    	學生     & :\studentCnName{}  \\
        指導教授  & :\advisorCnName{} \hspace{0.1cm} 教授 \\[0.5cm]
    \end{tabular}
    \end{singlespace}

    國立交通大學資訊科學與工程研究所碩士班 \\[0.5cm]
    \makebox[4em][s]{摘要} \\[0.5cm]
    	
  \end{center}
  \normalsize 
  \hspace{0.6cm} 在程式自動化測試與分析上,符號執行(symbolic execution)是目前經常被使用的一種方法,由於符號執行會紀錄並模擬出程式執行時的所有可能路徑,其數量會以指數的數量級不停成長,最終耗盡所有運算資源,這個問題被稱為路徑爆炸問題(path explosion problem);因此我們需要在有限的資源內採取某些策略來優先模擬較有價值的路徑,在本篇論文中我們提出使用以蒙地卡羅搜尋樹為基礎的搜尋策略來解決這個問題,並比較它與其他傳統策略如深度優先搜尋、廣度優先搜尋的效率。
  \\[0.7cm]
  關鍵字:Monte Carlo Tree Search(MCTS), Upper Confidence Bounds for Trees (UCT), symbolic execution
\end{titlepage}
\begin{titlepage}
  \begin{center}
    \LARGE
    \begin{singlespace}
  	 \textbf{\englishTitle{}} \\[0.5cm]
    \end{singlespace}
    
    \begin{singlespace}
    \begin{tabular}{r l}
    	Student     & : \studentEnName{}  \\
        Advisor  & : Dr. \advisorEnName{} \\[0.5cm]
    \end{tabular}
    \end{singlespace}
	
    \begin{singlespace}
    Institute of Computer Science and Engineering National Chiao Tung University\\[0.5cm]
    \end{singlespace}
    \textbf{ABSTRACT} \\[0.5cm]
    	
  \end{center}
  \normalsize 
  \hspace{0.6cm} Symbolic execution is a technology that is often used today in program automation testing and analysis. Since symbol execution traces and simulates all possible paths when a program executes, its number grows exponentially. This problem is called the path explosion problem. Therefore, we need to take some strategies within the limited resources to give priority to the more valuable path. In this paper, we propose to use The Carlo search tree-based search strategy solves this problem and compares it with other classic strategies such as depth-first search (DFS) and breadth-first search (BFS).
  \\[0.7cm]
  Keywords:Monte Carlo Tree Search(MCTS), Upper Confidence Bounds for Trees (UCT), symbolic execution
\end{titlepage}

\tableofcontents
\listoffigures
\listoftables

% 9/30
\chapter{Introduction} \pagenumbering{arabic} % enabling page numbering

在現今的軟體開發中,程式碼數量動輒數萬行,系統的複雜度也越來越高,傳統驗證程式正確性的方式如unit testing、code review等等...受限於人工而相當有限;在硬體計算能力突飛猛進的現代,自動化測試的方式又逐漸成為顯學,如fuzzing、symbolic execution...能夠自動尋找程式中可能的漏洞,其中symbolic execution是一種模擬執行的方法,它將程式的使用者輸入視為符號,並把程式執行過程和分支條件轉換為限制式,藉由求解限制式來獲得欲執行該路徑所需的使用者輸入為何;由於symbolic execution在遇到分支時會複製出一條新的路徑,兩條路徑分別探索執行if時和執行else時的狀態,因此路徑數量會以指數的數量級成長,造成探索路徑需要花費巨大的運算資源,這個問題被稱為path explosion problem,這篇論文欲以蒙地卡羅樹搜尋演算法來找出探索價值較高的路徑,使得symbolic execution能使用較短的時間內獲得更高的程式執行覆蓋率。

蒙地卡羅樹搜尋(Monte Carlo tree search)被廣泛的運用在遊戲人工智慧中,例如西洋棋、黑白棋、圍棋等等的棋盤遊戲,在2016年AlphaGo(一個結合蒙地卡羅樹搜尋和深度學習的圍棋AI程式)擊敗世界棋王後,更是一度引起相當多的討論;此演算法主要的精神在於對一棵樹,選擇子結點並嘗試用模擬的方式估計該結點的價值。和symbolic execution相當類似的地方在於同為需要探索一棵樹,而且要避免探索一些我們不感興趣或是沒有必要探索的路徑,例如無窮迴圈的狀況;我們認為結合蒙地卡羅樹搜尋做為symbolic execution挑選路徑執行的演算法,相較於傳統的深度優先搜尋法或廣度優先搜尋法,能有較佳的搜尋效率。

% \cite{thesis}.

% 9/30
\chapter{Background}

在這個章節將簡要介紹Symbolic execution與其遇到的問題,還有蒙地卡羅樹搜尋演算法的流程和優點。

\section{Symbolic execution}

在本篇論文中所提及的Symbolic execution為Dynamic symbolic execution(又名為Concolic execution),首先由K. Sen\cite{sen2007concolic}提出,和基於其想法實作的程式DART\cite{godefroid2005dart}、CUTE\cite{sen2005cute},而其近年又分為兩種類型,需要程式原始碼的code-based symbolic execution,如KLEE\cite{cadar2008klee}和不須程式碼而直接分析程式執行檔的binary-based symbolic execution如S2E\cite{chipounov2012s2e}、Mayhem\cite{cha2012mayhem};symbolic execution engines在分析程式時會將其載入並先轉換成一個共通的表示方式,接著會將使用者輸入(例如標準輸入、檔案、命令列參數)標記為symbolic變數,接著模擬程式的執行過程,將執行過程中遇到的程式碼轉換為數學邏輯限制式的形式,當在模擬的過程中遇到分支條件時,便複製出一條新的路徑,分別追蹤該分支條件為真和為假時的情況;當模擬執行結束時,把先前蒐集的邏輯限制式利用SMT solver\cite{vanegue2012smt}求解,以取得欲執行該路徑所需的實際輸入值;透過這個流程理論上我們可以追蹤所有的執行路徑,探索是否有不當的輸入值能觸發特定程式區段,或避開某些程式區段。

\section{Path Exploration}

雖然Symbolic execution理論上能夠探索所有執行路徑,但在探索的過程中可能會遭遇到一些問題:當分支條件取決於symbolic變數時,很有可能兩邊的條件都有可能成立,這意謂著程式必須複製並分別維護兩條路徑,直到發現該路徑無解為止,當路徑被不斷的複製就產生了路徑爆炸問題(path explosion problem)。

\section{Monte Carlo tree search}

\chapter{Design}
  
\chapter{Implementation}

\chapter{Evaluation}

% 9/30
\chapter{Related Work}

在\cite{sharma2012critical}\cite{schwartz2010all}中對symbolic execution所做的概述,提及了幾項目前的挑戰,其中最大的問題就是路徑選擇問題(path selection problem),當路徑爆炸問題已成一個不可避免的問題,目前已經提出的幾種解決方式有:
\begin{itemize}
\item \textbf{深度優先搜尋法} 
優先探索最深的路徑,在KLEE\cite{cadar2008klee}中便使用這種作法,但很有可能被困在無限迴圈而進行了無效的探索。
\item \textbf{concolic testing}
在\cite{sen2007concolic}中提出,除了以symbolic execution執行,也會使用具體的輸入真正執行程式來蒐集執行路徑,如\cite{sen2005cute}就使用了這種作法。
\item \textbf{隨機挑選路徑} KLEE\cite{cadar2008klee}也支援隨機挑選路徑的方式來避免進行了無效的探索。

\end{itemize}


\chapter{Conclusion}

\newpage

\printbibliography[title={References}]

\end{document}
